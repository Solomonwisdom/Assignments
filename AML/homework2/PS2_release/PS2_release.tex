\documentclass[a4paper,UTF8]{article}
\usepackage{ctex}
\usepackage[margin=1.25in]{geometry}
\usepackage{color}
\usepackage{graphicx}
\usepackage{amssymb}
\usepackage{amsmath}
\usepackage{amsthm}
\usepackage{enumerate}
\usepackage{bm}
\usepackage{hyperref}
\usepackage{epsfig}
\usepackage{color}
\usepackage{mdframed}
\usepackage{lipsum}
\usepackage{mathtools}
\usepackage{algorithm}
\usepackage{algorithmic}
\newmdtheoremenv{thm-box}{myThm}
\newmdtheoremenv{prop-box}{Proposition}
\newmdtheoremenv{def-box}{定义}

\setlength{\evensidemargin}{.25in}
\setlength{\textwidth}{6in}
\setlength{\topmargin}{-0.5in}
\setlength{\topmargin}{-0.5in}
% \setlength{\textheight}{9.5in}
%%%%%%%%%%%%%%%%%%此处用于设置页眉页脚%%%%%%%%%%%%%%%%%%
\usepackage{fancyhdr}                                
\usepackage{lastpage}                                           
\usepackage{layout}                                             
\footskip = 10pt 
\pagestyle{fancy}                    % 设置页眉                 
\lhead{2018年秋季}                    
\chead{高级机器学习}                                                
% \rhead{第\thepage/\pageref{LastPage}页} 
\rhead{作业二}                                                                                               
\cfoot{\thepage}                                                
\renewcommand{\headrulewidth}{1pt}  			%页眉线宽,设为0可以去页眉线
\setlength{\skip\footins}{0.5cm}    			%脚注与正文的距离           
\renewcommand{\footrulewidth}{0pt}  			%页脚线宽,设为0可以去页脚线

\makeatletter 									%设置双线页眉                                        
\def\headrule{{\if@fancyplain\let\headrulewidth\plainheadrulewidth\fi%
\hrule\@height 1.0pt \@width\headwidth\vskip1pt	%上面线为1pt粗  
\hrule\@height 0.5pt\@width\headwidth  			%下面0.5pt粗            
\vskip-2\headrulewidth\vskip-1pt}      			%两条线的距离1pt        
 \vspace{6mm}}     								%双线与下面正文之间的垂直间距              
\makeatother  

%%%%%%%%%%%%%%%%%%%%%%%%%%%%%%%%%%%%%%%%%%%%%%
\numberwithin{equation}{section}
%\usepackage[thmmarks, amsmath, thref]{ntheorem}
\newtheorem{myThm}{myThm}
\newtheorem*{myDef}{Definition}
\newtheorem*{mySol}{Solution}
\newtheorem*{myProof}{Proof}
\newcommand{\indep}{\rotatebox[origin=c]{90}{$\models$}}
\newcommand*\diff{\mathop{}\!\mathrm{d}}

\usepackage{multirow}

%--

%--
\begin{document}
\title{高级机器学习\\
作业二}
\author{学号, 作者姓名, 邮箱}
\maketitle

\section{[30pts] Learning Theory}
\begin{enumerate}[(1)]
	\item \textbf{[10pts] VC维} 

	试讨论最近邻分类器假设空间的VC维大小,并给出证明.
	\item \textbf{[10pts] Rademaher复杂度}
	
	试证明: 常数函数$c$的Rademaher复杂度为$0$.
	\item \textbf{[10pts] PAC} 
	
	$\mathcal{X}=\mathbb{R}^2, \mathcal{Y}= {0,1}.$假设空间$\mathcal{H}$定义如下:$\mathcal{H}=\{h_r:r \in \mathbb{R}_+\}$,其中$h_r (x)=\mathbb{I}(\parallel x \parallel \leq r)$,假定假设空间是可分的,证明$\mathcal{H}$是PAC可学习的,并且样本复杂度为$\frac{log(1/\delta)}{\epsilon}$
	\newline
	(提示:可考虑返回与训练集一致的最小圆的算法)
\end{enumerate}
\begin{myProof}
此处用于写证明(中英文均可)


\end{myProof}
\newpage

\section{[30pts] 文档主题模型}
在一个新闻数据集上实现文档主题模型(Latent Dirichlet Allocation~(LDA))~\cite{DBLP:journals/jmlr/BleiNJ03}.

我们提供了一个包含8,888条新闻的数据集,请在该数据集上完成LDA算法的使用及实现。

\begin{itemize}
	\item 数据集下载:\href{http://lamda.nju.edu.cn/ml2018grad/dataset/news.txt.zip}{新闻数据集.}
	\item 格式:每行是一条新闻.
\end{itemize}

数据预处理提示:你可能需要完成分词及去掉一些停用词等预处理工作.

\begin{enumerate}[(1)]
	\item \textbf{[10pts] 任务\#1:使用LDA模型} 
	\subitem A. 选择开源的LDA库(例如:\href{https://scikit-learn.org/stable/modules/generated/sklearn.decomposition.LatentDirichletAllocation.html}{scikit-learn}),并在提供的数据集上学习使用.
	\subitem B. 给出$K=\{5,10,20\}$个主题时,每个主题下概率最大的$M=10$个词及其概率.
	\item \textbf{[20pts] 任务\#2:实现LDA模型} 
	\subitem A. 不借助开源库,自己完成LDA算法.
	\subitem B. 给出$K=\{5,10,20\}$个主题时,每个主题下概率最大的$M=10$个词及其概率.
\end{enumerate}
\begin{mySol}

\end{mySol}
\newpage


\section{[40pts] 强化学习实验}
用DQN (deep Q Networks) 训练Flappy Bird. 请各位同学根据DQN算法流程,补全提供的代码包中\text{deep\_q\_networkd.py}文件中“\# TODO”部分代码 (补全 epsilon-greedy action selection以及Q learning updating),了解DQN算法,并进行训练,本实验时间相对较久.

本次实验所需要的依赖如下:
\begin{itemize}
	\item python2.7 or python3;
	\item pygame;
	\item OpenCV-python;
	\item TensorFlow (建议使用1.1-1.6).
\end{itemize}

强化学习中经典的off-policy算法Q-Learning的原始版本采用表格形式来记录$Q$函数,显然只能应用于有限离散状态、有限离散动作且状态、动作数量较少的情况下,即有维度灾难问题 (表格大小正比于 $|S|*|A|$). 采用函数近似法,假定$Q$函数可由状态特征经过某个函数的映射到对应动作的评价值上,可扩大Q-Learning使用范围. 近年来,DeepMind结合深度模型强大的表达能力,用深度神经网络作为近似函数来表达强化学习中的 $Q$ 函数,进一步扩大了Q-Learning可用范围. DQN中采用experience replay和target network两种技术,使DQN的训练更加高效且鲁棒,并在atari的部分游戏上取得了人类水平的表现.

DQN的流程大致如下~\ref{alg:DQN}:
\begin{algorithm}
	\caption{DQN with experience replay}
	\label{alg:DQN}
	\begin{algorithmic}
		\STATE {Initialize replay memory $D$ to capacity $N$} 
		\STATE {Initialize action-value function $Q$ with random weights $\theta$}
		\STATE {Initialize target action-value function $\hat{Q}$ with weights $\theta^- =\theta$}
		\FOR{$episode=1,M$}
		\STATE {Initialize sequence $s_1={x_1}$ and preprocessed sequence $\phi_1=\phi(s_1)$}
		\FOR{$t=1,T$}
		\STATE {With probability $\epsilon$ select a random action $a_t$}
		\STATE {otherwise select $a_t=\arg\max_a Q(\phi(s_t),a;\theta)$}
		\STATE {Execute action $a_t$ in emulator and observe reward $r_t$ and image $x_{t+1}$}
		\STATE {Set $s_{t+1}=s_t,a_t,x_{t+1}$ and preprocess $\phi_{t+1}=\phi(s_{t+1})$}
		\STATE {Store transition $(\phi_t,a_t,r_t,\phi_t)$ in $D$}
		\STATE {Sample random minibatch of transitions $(\phi_j,a_j,r_j,\phi_{j+1})$ from $D$}
		\STATE {Set 
			\begin{equation}
			f(x)=
			\begin{cases}
			r_j& \text{if episode terminates at step $j+1$}\\
			r_j+\gamma\max_{a'}\hat{Q}(\phi_{j=1},a';\theta^-)& \text{otherwise}
			\end{cases}
			\end{equation}}
		\STATE {Perform a gradient descent step on $(y_j-Q(\phi_j,a_j;\theta))^2$ with respect to the network parameters $\theta$}
		\STATE {Every $C$ steps reset $\hat{Q}=Q$}
		\ENDFOR
		\ENDFOR
	\end{algorithmic}
\end{algorithm}

上图是15年DeepMind发表在Nature上文章中所采用的算法流程,包含了 experience replay和target network技术,本次实验不要实现target network ,仅需要实现experience replay即可 (实现target network可额外获得$5$pts bonus). 感兴趣的同学可参阅DQN相关教程或文章,进一步了解两种技术. 

本次实验中状态太输入为raw pixel,转为$80*80$的灰度图 (采用openCV转换),并将历史最近$3$个frame叠加到当前frame中作为状态输入,即每一步输入状态为$4*80*80$,动作为$2$维离散动作 (上、下,action为$2$维one-hot编码). 网络模型已经搭建好 (采用TensorFlow 搭建),输入为$4*80*80$,输出为$2$,对应每个动作对应的 $Q$ 值。如下图所示~\ref{fig:rl_1}.
\begin{figure}[!htb]
	\centering
	\includegraphics[scale=0.4]{rl_1.png}
	\caption{网络模型.}
	\label{fig:rl_1}
\end{figure}

游戏环境中,单步奖励为$0.1$,越过一个管道$+1$,死亡得到$-1$的惩罚. 可采用其他深度学习框架,如 pytorch、keras 等搭建模型并完成训练代码. DQN算法设置可采用如下配置:
\begin{itemize}
	\item GAMMA = 0.99 \# decay rate of past observations;
	\item OBSERVE = 10000. \# timesteps to observe before training;
	\item EXPLORE = 2000000. \# frames over which to anneal epsilon;
	\item FINAL\_EPSILON = 0.0001 \# final value of epsilon;
	\item INITIAL\_EPSILON = 0.1~0.2 \# starting value of epsilon;
	\item REPLAY\_MEMORY = 50000 \# number of previous transitions to remember;
	\item BATCH = 32 \# size of minibatch;
	\item FRAME\_PER\_ACTION = 1.
\end{itemize}

默认一直训练不会终止,每$10,000$ frames保存一个模型,默认最大保存$5$个,保存的模型可恢复用来测试,默认保存在save\_model 目录下. 采用GPU可加速训练,仅使用双核CPU训练时,采用如上配置,总样本量到$1$M ($1,000,000$个state) 需要时间为$20$~$24$h,大概$3$M可训练出相当不错的策略,考虑到计算咨询和时间,可自行选择训练量.

采用其他深度学习框架时,只需要保持从环境中获得返回的状态、奖励信息,以及是否终止,并可在环境中执行action (再次注意,action为$2$维one-hot编码). Agent与环境交互过程如下所示:
\begin{itemize}
	\item sys.path.append("game/");
	\item import wrapped\_flappy\_bird as game \# import game environment;
	\item game\_state = game.GameState() \# initialize;
	\item \# execute an action and get info from the environment;
	\item $x_t, r_0$, terminal = game\_state.frame\_step(action).
\end{itemize}

本实验提交要求:

仅需提供补全后deep\_q\_network.py文件,以及训练后的短视频 (连续飞行$5-10$s 即可) 或图片或gif动图等辅助证明材料,并说明训练使用样本量. 如果有任何修改或补充说明,请一并说明. (建议写Readme文件或报告)



\begin{mySol}
此处用于写解答(中英文均可)

\end{mySol}
\newpage


\bibliographystyle{plain}
\bibliography{ref}
\end{document}